\documentclass[letterpaper]{article} 
\usepackage[left = 0.5in, right = 0.5in, top = 0.9in, bottom = 0.9in]{geometry}
\usepackage{enumitem}
\usepackage{multicol}
\usepackage[spanish]{babel}
\usepackage[utf8]{inputenc}

\usepackage{amsmath,amssymb,amsthm}
\usepackage{tikz-cd}
\usepackage{mathrsfs}
\usepackage[bbgreekl]{mathbbol}
\usepackage{dsfont}
\usepackage{graphicx}
\graphicspath{{img/}}

\newcommand{\N}{\mathbb N}
\newcommand{\Z}{\mathbb Z}
\newcommand{\Q}{\mathbb{Q}}
\newcommand{\I}{\mathbb{I}}
\newcommand{\R}{\mathbb{R}}
\newcommand{\C}{\mathbb{C}}
\newcommand{\F}{\mathcal{F}}
\newcommand{\G}{\mathcal{G}}
\newcommand{\B}{\mathcal{B}}
\newcommand{\abs}[1]{\left\lvert #1 \right\rvert}
\newcommand{\inv}{^{-1}}
\renewcommand{\to}{\rightarrow}
\newcommand{\ent}{\Longrightarrow}
\newcommand{\E}{\mathbb{E}}
\renewcommand{\P}{\mathbb{P}}
\newcommand{\1}{\mathds{1}}
\renewcommand{\qedsymbol}{$\blacksquare$}

\theoremstyle{definition}
\newtheorem{dfn}{Definición}
\theoremstyle{definition}
\newtheorem{teo}{Teorema}
\theoremstyle{definition}
\newtheorem{cor}{Corolario}
\theoremstyle{definition}
\newtheorem{prop}{Proposición}
\theoremstyle{definition}
\newtheorem{obs}{Observación}


\title{\textbf{}}
\author{Iván Irving Rosas Domínguez}
\date{\today}

\DeclareSymbolFontAlphabet{\mathbbm}{bbold}
\DeclareSymbolFontAlphabet{\mathbb}{AMSb}
\DeclareMathSymbol\bbDelta  \mathord{bbold}{"01}

\begin{document}
\maketitle

%\begin{abstract}
%\end{abstract}

\begin{itemize}
    \item[\textbf{1.}] Probar que la medida $\mu_0$, definida en el teorema de Daniell-Kolmogorov, es finita aditiva en los cilindros, i.e. en $\mathfrak{C}$ de acuerdo
    a la notación del curso.
    \item[\textbf{2.}] En la propiedad de Markov para el caso de cadenas de Markov, completar el argumento de clases monótonas.
    
    \item[\textbf{3.}] Probar que 
    \[
    \F_T=\left\{A\in\F \mid A\cap \{T=n\}\in \F_n \text{ para toda } n\in \N \right\}    
    \]
    es una $\sigma$-álgebra. Además, ver que $X_T\1_{\{T<\infty\}}$ es $\F_T$-medible.
    \item[\textbf{4.}] \textbf{(Propiedad de Markov Fuerte)} Sea $T$ un tiempo de paro finito 
    casi seguramente, i.e. $\P_x\left(T<\infty\right)=1$. Entonces bajo el evento $\{X_T=y\}$, la cadena trasladada 
    $X\circ\Theta_T$ es independiente de $\F_T$ y tiene por ley $\P_y$.
    \item[\textbf{5.}] \textbf{(Cambio de tiempo)} El objetivo de este ejercicio es el de 
    realizar una transformación de cambio de tiempo a una cadena de Markov y ver que transforma 
    a una cadena de Markov en otra cadena (con matriz de transición diferente).
    \begin{enumerate}
        \item Caminata aleatoria (continua por la izquierda): Sean $\Pi$ una medida de probabilidad
        en $\{-1,0,1,2,...\}$ y $(\xi_n)_{n\geq1}$ una sucesión de v.a.i.i.d. con ley común $\Pi$. Para $x\in \Z$, 
        vamos a denotar por $\P_x$ a la ley de $S=(S_n,n\geq0)$, donde 
        \[
            S_n=S_0+\sum_{j=1}^{n}\xi_j \quad n\geq1, \quad \text{ y } \quad \P_x(S_0=x)=1. 
        \]
        Además vamos a definir a $T_0:=\inf \left\{n\geq0 \mid S_n=0\right\}$. Probar que $T_0$ es un 
        tiempo de paro.
        \item Procesos de Galton-Watson: Sea $\gamma$ una medida de probabilidad en $\{0,1,...\}$, la cual 
        llamaremos ley de reproducción (ley del número de hijos de un individuo). Para $x\in \N$, vamos a denotar por 
        $\Q_x$ a la ley del proceso de Galton-Watson $Z=\left(Z_n,n\geq0\right)$ que empieza con $x$ individuos, 
        donde 
        \[
        Z_0=x, \qquad \Q_x-c.s., \qquad Z_{n+1}=\sum_{j=1}^{Z_n}\alpha_{n,j},    
        \]
        y $\alpha_{n,j}$ es el número de hijos del $j$-ésimo individuo de la generación $n$. 
        Vamos a suponer que $\{\alpha_{n,j},n\geq0,j\geq1\}$ son v.a.i.i.d. con ley común $\gamma$. Probar 
        que $Z$ es una cadena de Markov con matriz de transición 
        \[
        Q(x,y)=\begin{cases}
            1 & \text{si } x=y=0\\
            0 & \text{si } x=0\neq y\\
            \gamma^{*x}(y) & \text{en otro caso,}
        \end{cases}    
        \]
        donde $\gamma^{*x}$ no es más que la convolución de $\gamma$, $x$-veces y representa la ley 
        de $\alpha_{0,1}+...+\alpha_{0,n}$.
        \item Supongamos que $x>0$. Vamos a definir de manera inductiva a $\tau_{n+1}:=\tau_{n}+S_{\tau_n}$ con 
        $\tau_0=0$. Verificar que $\tau_n$ es un tiempo de paro en la filtración natural de $S$, que $\tau_n\leq T_0$ y 
        que $\lim_{n\to \infty}\tau_n=T_0$.
        \item Sea $Z_n:=S_{\tau_n}$ y $\Q_x$ la ley de $Z=(Z_n,n\geq0)$ bajo $\P_x$. Verificar 
        que $\Q_x$ es la ley de un proceso de Galton-Watson y expresar a la ley de reproducción $\gamma$ en términos de $\Pi$. Además 
        verificar que la población total $\sum_{n=0}^{\infty}X_n$, bajo $\Q_x$, tiene la misma 
        ley que $T_0$ bajo $\P_x$.
    \end{enumerate}
    \item[\textbf{6.}] Probar que el único conjunto $\F$-medible contenido en $\mathcal{C}\left([0,\infty),\R^{d}\right)$ es el conjunto 
    vacío y deducir que $\mathcal{C}\left([0,\infty),\R^d\right)$ no es $\F$-medible.
    \item[\textbf{7.}] Sean $\widetilde{X}$ y $X$ dos procesos continuos casi seguramente. Probar que si 
    $\widetilde{X}$ es una modificación de $X$, entonces $\widetilde{X}$ y $X$ son indistinguibles.
    \item[\textbf{8.}] Probar que existe una única medida $\mathbb{W}$ en $\left(\mathcal{C}\left([0,\infty),\R^d\right),\B(\mathcal{C})\right)$,
    donde $\B(\mathcal{C})$ es la $\sigma$-álgebra de Borel con la topología de los compactos-abiertos, tal que el proceso canónico $X=(X_t,t\geq0)$ 
    definido por 
    \[
    X_t(\omega)=\omega(t), \qquad \text{ donde }\omega\in \mathcal{C}\left([0,\infty),\R^d\right),    
    \]
    es un movimiento browniano $d$-dimensional en $\left(\mathcal{C}\left([0,\infty),\R^d\right),\B(\mathcal{C}),\mathbb{W}\right)$.

\end{itemize}

\end{document}